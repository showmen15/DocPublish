\chapter{Koordynacja ruchu robotów mobilnych}

Czym jest koordynacja ruchu robotów mobilnych.

Koordynacja problem klasy NP.

Pojedynczy robot, a grupa robotów.

Dążenie do wykonania powierzonego zadania.


\section{Problemy koordynacji ruchu robotów mobilnych}

Jakie problemy należy rozwiązać.

Ograniczenia fizyczne robotów.

Problem wzajemnego postrzegania się robotów - czujniki. 

Problemy komunikacyjne - czyli jak powiedzieć gdzie jestem i co chce zrobić.

Unikanie kolizji, zakleszczeń, bezpieczeństwo ruchu wynikające z dostępnej przestrzeni. 

Sterowanie grupą robotów.

Planowanie ścieżki robota - dla jednego oraz wielu robotów - złożony problem 

Planowanie ruchu ograniczenia kinematyczne i dynamiczne

Współpraca robot w celu wykonania konkretnego zadania 

Kontrola formacji robotów - czy jest oraz jak ją kontrolować, re-konfiguracja formacji robotów

Kontrola łańcucha robotów.


\section{Przegląd istniejących algorytmów koordynacji ruchu robotów mobilnych}

Podział algorytmów koordynacji ruchu według kryteriów: ze względu na interakcje pomiędzy robotami, sposób podejmowania decyzji.

Dwa diagramy według których zostanie dokonany podział.

Diagram przedstawiający podział algorytmów koordynacji ze względu na interakcje między robotami.

Diagram przestawiający podział algorytmów koordynacji ze względu na sposób podejmowania decyzji.

Dla każdej z metod opisujemy wady i zalety oraz jakie efekty uzyskali autorzy.


\subsection{Scentralizowane}

charakterystyka + przegląd rozwiązań 

\subsection{Z planowaniem początkowym}

charakterystyka + przegląd rozwiązań 

\subsection{Z planowaniem dynamicznym}

charakterystyka + przegląd rozwiązań 

\subsection{Zdecentralizowane}

Metody zdecentralizowane (rozproszone)  - charakterystyka 

charakterystyka + przegląd rozwiązań 

\subsection{Z jawną komunikacją}

charakterystyka + przegląd rozwiązań 

\subsection{Bez jawnej komunikacji}

charakterystyka + przegląd rozwiązań 


\subsection{Reaktywne}

behawioralne
charakterystyka + przegląd rozwiązań 

\subsection{Proaktywne}

celowe
charakterystyka + przegląd rozwiązań 

\subsection{Rozłączne}

charakterystyka + przegląd rozwiązań 

\subsection{Opere o priorytety}

charakterystyka + przegląd rozwiązań 



Koordynacja ruchu inspirowana zjawiskami społecznymi - ??? może są jakieś artykuły o tym 

??? W jaki sposób dokonać podziału metod??

??? Jak dzielić zebrane materiały.