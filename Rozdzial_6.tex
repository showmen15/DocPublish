\chapter{Eksperymenty}

środowisko testowe 
Cel

Opis labiryntu - przykładowa map - co zawiera i dla kogo jest dostępna (symulator robot)

\section{Eksperymenty symulacyjne}

Opis wykonanych przypadków testowych.

Lokalizacja robota w na mapie - w symulacji.

Symulowane z odstępem czasu 200ms.

W jaki sposób uruchamiamy, jak mierzymy, kiedy uznajemy że przypadku nie da się policzyć - po jakim czasie i przy jakich warunkach.

komputer wykorzystany do symulacji - jago parametry
Opis środowiska symulacyjnego
Schemat symulatora z jakich bloków się składa
jak jest konfigurowany 
baza danych 


Dla każdego z podrozdziałów następująca struktura poszerzona o dodatkowe elementy:
Opis scenariusza testowego sposobu ustawiania robotów.
Schemat prezentujący ustawienie robotów.
Prezentacja wyników, wraz z omówieniem.

Od 1-1 do 10-10

\subsection{Otwarta przestrzeń}

\subsubsection{Grupa na grupę}

\subsubsection{Ustawienie w okrąg}

\subsubsection{Ustawienie skos}

\subsection{Przejście przez drzwi}

Ustawienie robotów jaki ma wpływ na wykonanie zadania - jeżeli straszniejsza grupa chce wejść lub wyjść jak to się przekłada na wyniki.
Wpływ strefy wejściowo-wyjściowej na poszczególne roboty 

\subsection{Wąski korytarz}

Problem wzajemnego przepychania się robotów.
Inny sposób zarządzania taka grupą - wniosek.

\subsection{Skrzyżowanie równorzędne}

Algorytm WR przy doborze bazowego FF upraszcza się  do algorytmu prawej dłoni (szczególny przypadek)

\subsection{Skrzyżowanie typu 8}

\subsection{Mijanka}

Prezentacja zakleszczeń wynikających z wad metod RVO, prawa dłoń oraz WR, CP (ostanie dwa zależą od doboru bazowego FF)

\subsection{Podsumowanie}

Wstępne wnioski z przeprowadzonych eksperymentów.

\section{Eksperymenty na rzeczywistych robotach}

Możemy w prosty sposób przejść z symulatora na roboty fizyczne.

Platforma wykorzystana w eksperymentach.

\subsection{Platforma CAPO}

Nawiązanie do naszej konstrukcji - CAPO - platformy którą w łatwy sposób możemy dostosowywać do aktualnych potrzeb z czego się składa co umożliwia 

Model ruchu napęd różnicowy robota wraz w wymiarami.

Architektura platformy sprzętowo-programowa, zdjęcia. 

Interface programowania robota

Specyfikacja sprzętowe.

Schemat blokowy podzespołów.

Wykorzystywane elementy.

Lokalizacja robota w labiryncie - system lokalizacji robota - schemat 

System rozgłaszania komunikatów o pozycji - schemat

\subsection{Scenariusze testowe}


Opis wykonanych przypadków testowych.

Dla każdego z podrozdziałów następująca struktura poszerzona o dodatkowe elementy:



Opis scenariusza testowego sposobu ustawiania robotów.
Schemat prezentujący ustawienie robotów.
Prezentacja wyników, wraz z omówieniem.

Uzasadnienie dlaczego wybraliśmy takie przypadki i dlaczego nie wybraliśmy innych.
 
\subsection{Otwarta przestrzeń}

\subsubsection{Grupa na grupę}

1-1, 1-2

\subsubsection{Ustawienie skos}

2-2

\subsection{Przejście przez drzwi}

1-1, 1-2

\subsection{Mijanka}

1-1

\subsection{Podsumowanie}

Wstępne wnioski z przeprowadzonych eksperymentów.
