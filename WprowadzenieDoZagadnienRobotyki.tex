\chapter{Wprowadzenie do zagadnień robotyki i programowania robotów}

\section{Roboty podział i charakterystyka}

Roboty jakie są - krótkie wprowadzenie.  

Schemat z podziałem: modele robotów - podział jakie mamy roboty kroczące, latającą, pływające, jeżdżące 


Roboty kroczące

Krotki opis kroczących charakterystyka wraz z przykładami, wady zalety.

Roboty latające

Krotki opis latających charakterystyka wraz z przykładami, wady zalety.

Roboty pływające

Krotki opis pływających charakterystyka wraz z przykładami, wady zalety.


\section{Roboty kołowe}

Rozbudowana charakterystyka wraz z przykładami, wady zalety.

Platformy do jazdy w taranie - roboty wyposażone w gąsienice, koła terenowe, koła typy meccanum;.

W co są wyposażone - czujniki, manipulatory, 

Do czego możemy je wykorzystywać - patrolowanie, systemy magazynowe, roboty towarzyszące (zabawa), informacyjne (promujące dany produkt), roboty ratunkowe, 

Wzmianka o naszej platformie CAPO - szerszy opis w rozdziale Eksperymenty na rzeczywistych robotach


\section{Autonomia robotów}

Sposoby sterowania - przez operatora, częściowo autonomiczne, w pełni autonomiczne.

Co rozumiemy pod pojęciem że robot jest autonomiczny - jakie powinien spełniać kryteria.

Autonomia zasilania itd.

Podejmowania decyzji i wpływ na środowisko.

Autonomia starowania.

Wypracowanie pewnego zestawu metod umożliwiającego nam kontrolowanie pojedynczego robota jak i również grypy robotów - algorytm koordynacji robotów.


